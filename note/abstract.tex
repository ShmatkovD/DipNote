\sectioncentered*{Реферат}
\thispagestyle{empty}

% Зачем: чтобы можно было вывести общее число страниц.
% Добавляется единица, поскольку последняя страница -- ведомость.
\FPeval{\totalpages}{round(\getpagerefnumber{LastPage}, 0)}

\begin{center}
	Пояснительная записка \num{\totalpages}~с., \num{\totfig{}}~рис., \num{\tottab{}}~табл., \num{\toteq{}}~формулы, \num{\totref{}}~источника.
	\MakeUppercase{Программное средство, веб-сервис, университет, анализ музыкальных композиций, машинное обучение, индивидуальные задания}
\end{center}

Цель настоящего дипломного проекта состоит в разработке программной системы, предназначенной для автоматизации извлечения характеристик музыкальных произведений.

В процессе анализа предметной области были выделены основные характеристики музыкальных композиций и способы их извлечения, которые в настоящее время практически не охвачены автоматизацией. Было проведено их исследование и моделирование. Кроме того, рассмотрены существующие средства, разрозненно разрабатываемые и применяемые сотрудниками университетов и отдельными компаниями (так называемые частичные аналоги). Выработаны функциональные и нефункциональные требования.

Была разработана архитектура программной системы, для каждой ее составной части было проведено разграничение реализуемых задач, проектирование, уточнение используемых технологий и собственно разработка. Были выбраны наиболее современные средства разработки, широко применяемые в индустрии.

Полученные в ходе технико-экономического обоснования результаты о прибыли для разработчика, пользователя, уровень рентабельности, а также экономический эффект доказывают целесообразность разработки проекта.
