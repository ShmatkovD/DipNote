\section{Руководство по установке и использованию}
\label{sec:manual}

\subsection{Установка}
\label{sec:manual:install}

Для того, чтобы установить и подготовить сервис для использования, необходимо установить Docker на используемую систему.

Далее мы опишем процесс установки Docker на системы семейства Linux/Debian.

Для начала обновим базу данных пакетов:

\begin{lstlisting}
sudo apt-get update
\end{lstlisting}

Теперь установим Docker. Добавьте ключ GPG официального репозитория Docker в вашу систему:

\begin{lstlisting}
sudo apt-key adv --keyserver hkp://p80.pool.sks-keyservers.net:80 --recv-keys 58118E89F3A912897C070ADBF76221572C52609D
\end{lstlisting}

Добавим репозиторий Docker в список источников пакетов утилиты APT:

\begin{lstlisting}
sudo apt-add-repository 'deb https://apt.dockerproject.org/repo ubuntu-xenial main'
\end{lstlisting}

Обновим базу данных пакетов информацией о пакетах Docker из вновь добавленного репозитория:

\begin{lstlisting}
sudo apt-get update
\end{lstlisting}

Далее, наконец-то, установим Docker:

\begin{lstlisting}
sudo apt-get install -y docker-engine
\end{lstlisting}

После завершения выполнения этой команды Docker должен быть установлен, демон запущен, и процесс должен запускаться при загрузке системы. Проверим, что процесс запущен:

\begin{lstlisting}
sudo systemctl status docker
\end{lstlisting}

Вывод должен быть похож на представленный ниже, сервис должен быть запущен и активен:

\begin{lstlisting}
docker.service - Docker Application Container Engine
Loaded: loaded (/lib/systemd/system/docker.service; enabled; vendor preset: enabled)
Active: active (running) since Sun 2016-05-01 06:53:52 CDT; 1 weeks 3 days ago
  Docs: https://docs.docker.com
Main PID: 749 (docker)
\end{lstlisting}

При установке Docker мы получаем не только сервис (демон) Docker, но и утилиту командной строки docker или клиент Docker.

Далее необходимо загрузить контейнер на машину:

\begin{lstlisting}
docker pull dshmatkov/genrify
\end{lstlisting}

Перед тем как загружать контейнер, необходимо авторизоваться у автора контейнера, так как он является приватным.

После нужно запустить контейнер:

\begin{lstlisting}
docker run -p 127.0.0.1:80:8000 dshmatkov/genrify
\end{lstlisting}

Порт контейнера 8000 будет привязан к порту 80 по адресу 127.0.0.1 на локальной системе.

Далее можно приступить к использованию сервиса.

\subsection{Использование}
\label{sec:manual:usage}

Для того, чтобы получать характеристики музыкальных композиций, необходимо пересылать файлы аудиозаписей в теле HTTP-запроса в контейнер на порт 8000.

В результате вы получите ответ от сервиса, который будет содержать искомые характеристики.

Для того, чтобы определить формат, в котором вы хотите получать ответ от сервиса, необходимо использовать заголовок Content-Type.

Для того, чтобы получать ответы в формате XML - необходимо выставить значение этого заголовка в application/xml.

Для того, чтобы получать ответы в формать JSON - необходимо выставить значение этого заголовка в application/json.
