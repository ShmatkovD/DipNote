\section{Обзор существующих аналогов}
\label{sec:analogues}

Существует ряд компаний и сервисов, которые решают задачу извлечения характеристик музыкальных композиций и задачи, достаточно близкие к данной. Рассмотрим наиболее популярные и известные из них.

\subsection{The Echo Nest}
\label{sec:analogues:ten}
The Echo Nest - компания из Соммервиля, которая занимается разработкой сервиса для анализа музыкальных композиций и составлении рекомендаций. Их продукт собирает информацию, используя акустический и текстовый анализ. Текстовый анализ состоит в том, что любое упоминание музыкальной композици, которое найдено в интернете, проходит через системы The Echo Nest. Эти системы настроены на то, чтобы извлекать ключевые слова и термины. Этим данным потом присвается собственный вес, который говорит об их важности. Акустический анализ начинается с того, что композиция разбивается на небольшие кусочки. Затем для каждого сегмента определяются громкость, тембр и другие характеристики. Далее полученные данные объединяются и анализируются. Анализ проводится с помощью методов машинного обучения, что позволяет понять композицию на высоком уровне. Объединение акустического и текстового анализа позволило создать мощную технологию, которая сделала компанию мировым лидером в алгоритмах анализа музыки. В 2014 году компанию купил мировой гигант стримминга музыки - Spotify.

\subsection{Niland}
\label{sec:analogues:niland}
Niland - компания из Парижа, которая занимается поиском и реккомендациями музыкальных композиций. Их продукт анализирует композицию, используя акустический анализ. Решаемые задачи разделяются на 2 типа: оценка похожести и классификация. Для оценки похожести используется подход music2vec. То есть преобразование композиции в вектор из значений характеристик. Для классификации композиций используется преобразование музыки в вектор для дальнейшей обработки с помощью алгоритмов машинного обучения. Преобразование музыки в вектор включает в себя:
\begin{enumerate}
    \item вычисление спектрограммы, т.е. вычисление интенсивности различных частот в различные моменты времени;
    \item извлечение кратковременных особенностей, т.е. свойств сигнала, \\которые имеют высокую частоту обновления;
    \item моделирование статического распределения кратковременных особенностей и объединение их в вектор.
\end{enumerate}

В результате получается вектор большой размерности, порядка тысячи элементов, который впоследствие используется для классификации композиций и извлечения характеристик высокого уровня.

\subsection{Music Technology Group}
\label{sec:analogues:mtg}
Music Technology Group - исследовательская группа с факультета информации и коммуникационных технологий университета Pompeu Fabra, \\Барселона. Специализируется на вычислительных задачах в области музыки и звука. Свои исследования группа основывает на знаниях из других областей, таких, как обработка сигналов, машинное обучение и взаимодействие человека с компьютером. Темы исследований, которыми группа занимается в данный момент:
\begin{enumerate}
    \item обработка аудиосигналов, т.е. спектральное моделирование для \\синтеза и трансформации звука;
    \item описание звука и музыки, т.е. семантический анализ и классификация аудио;
    \item продвинутое взаимодействие с музыкой, т.е. разработка интерфейсов для создания и изучения музыки;
    \item звуковые и музыкальные сообщества, т.е. технологии социальных сетей для музкальных и звуковых приложений.
\end{enumerate}

Группа создала библиотеку для определения степени похожести и \\классификации музыки, для описания музыки с помощью высокоуровневых характеристик - Gaia. Обрабатываются данные в несколько этапов: подготовка данных, обучение на подготовленных данных, проверка результатов.
Подготовка данных включает в себя:
\begin{enumerate}
  \item удаление метаданных;
  \item извлечение тональных характеристик;
  \item нормализацию выделенных характеристик;
  \item преобразование низкоуровневых характеристик в \\нормально распределенные величины.
\end{enumerate}

Для обучения используется метод опорных векторов с полиномиальным ядром (однородным). В качестве метрики точности используется н-крат\-ная кроссвалидация.

\subsection{DeepMind}
\label{sec:analogues:deepmind}
DeepMind - компания, которая занимается искусственным интеллектом. Ранее она была известна под названием Google DeepMind. Компания занимается решением проблем, связанных с интеллектом, т.е. занимается разработкой обучающихся алгоритмов общего назначения. На данный момент компания сосредоточила свои усилия на разработке интеллекта, который \\способен играть в компьютерные игры - от стратегических до аркад.

Сандер Дилеман, ученый из этой компании, стал соавтором статьи\cite{deepcontent}, в которой утвержадется, что глубинное обучение\cite{deep} способно гораздо лучше справляться с рекомендациями, чем колаборативная фильтрация. В качестве инструмента использовалась сверточная нейронная сеть с 7-8 слоями. Для визуализации данных использовался алгоритм t-SNE. Алгоритм определял интструменты, аккорды, и даже гармонии и прогрессии.
