\subsection{Обзор предметной области и литературы}
\label{sec:analysis:literature}

Извлечение музыкальной информации - небольшая, но интересная и быстрорастущая область исследований, которая охватывает большой диапазон продуктов, используемых по всему миру. Эта область охватывает несколько дисциплин научных знаний: музыковедение, психологию, обработку сигналов, машинное обучение и комбинации этих дисциплин. Задача, которую будет решать разрабатываемый сервис, относится к задачам извлечения музыкальной информации.

Несмотря на то, что извлечение музыкальной информации является еще небольшой областью исследований, для извлечения характеристик музыкальных приложений применяется достаточное количество методов. Эти методы различаются как на основаниии применяемых подходов, так и на основании уровней извлекаемых характеристик. На основании характеристик выделяют методы которые извлекают низкоуровневые характеристики (спектр, ритм) и методы которые извлекают высокоуровневые характеристики (жанр, настроение). В основном, нас будут интересовать методы которые извлекают высокоуровневые характеристики.

Для извлечения высокоуровнывых характеристик выделяют два класса методов: основанные на математических функциях методы, методы в основе которых лежит машинное обучение. В музыке появляются новые направления, а старые могут изменятся, поэтому наиболее перспективными являются методы, которые способны постраиваться под переменчивую природу музыки. Под такие условия в большей степени подходят методы, в основе которых лежит машинное обучение.

Машинное обучение - класс методов, характерной чертой которых является не прямое решение задачи, а решение, которое строится в процессе обучения при применении решения к множеству сходных задач.


ДОПИСАТЬ!!!!
