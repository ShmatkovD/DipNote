\subsection{Оценка экономической эффективности применения ПС у пользователя}
\label{sec:economics:effect}

\begin{longtable}{{
      |>{\raggedright}m{0.16\textwidth} |
	    >{\centering}m{0.135\textwidth} |
      >{\centering}m{0.135\textwidth} |
      >{\centering}m{0.135\textwidth} |
      >{\centering}m{0.135\textwidth} |
	    >{\centering\arraybackslash}m{0.135\textwidth} |
  }}
  \caption{Исходные данные для расчета экономического эффекта}
  \label{sec:economics:effect:tab_source}

  \hline
  {\begin{center} На\-име\-но\-ва\-ние показателей \end{center}} &
  {\begin{center} Обо\-зна\-че\-ния \end{center}} &
  {\begin{center} Единицы измерения \end{center}} &
  {\begin{center} Значения показателя в базовом варианте \end{center}} &
  {\begin{center} Значения показателя в новом варианте \end{center}} &
  {\begin{center} На\-име\-но\-ва\-ние источника информации \end{center}\arraybackslash} \\
  \endfirsthead

  \caption*{Продолжение таблицы \ref{sec:economics:effect:tab_source}}\\
  \hline
	\centering 1 & \centering 2 & \centering 3 & \centering 4 & \centering 5 & \centering\arraybackslash 6 \\
  \hline
  \endhead

  \hline
	\centering 1 & \centering 2 & \centering 3 & \centering 4 & \centering 5 & \centering\arraybackslash 6 \\

  \hline
  1 Капитальные вложения, включая затраты пользователя на приобретение &
  ${\text{К}}_{\text{ПР}}$ & руб. & & 126352 & Договор заказчика с разработчиком \\

  \hline
  2 Затраты на сопровождение ПО &
  ${\text{К}}_{\text{С}}$ & руб. & & 17132.53 & Договор заказчика с разработчиком \\

  \hline
  3 Время простоя сервиса, обусловленное ПО, в день &
  ${\text{П}}_{\text{1}}$,${\text{П}}_{\text{2}}$ & мин & 120 & 10 & Расчетные данные пользователя и паспорт ПО \\

  \hline
  4 Стоимость одного часа простоя &
  ${\text{С}}_{\text{П}}$ & руб. & 30.1 & 30.1 & Расчетные данные пользователя и паспорт ПО \\

  \hline
  5 Среднемесячная ЗП одного программиста &
  ${\text{З}}_{\text{СМ}}$ & руб. & 881.57 & 881.57 & Расчетные данные пользователя \\

  \hline
  6 Коэффициент начислений на зарплату &
  ${\text{К}}_{\text{Н}}$ & & 1.2 & 1.2 & Рас\-чи\-ты\-ва\-ет\-ся по данным пользователя \\

  \hline
  7 Среднемесячное количество рабочих дней &
  ${\text{Д}}_{\text{Р}}$ & день & 21 & 21 & Принято для расчета \\

  \hline
  8 Количество типовых задач, решаемых за год &
  ${\text{З}}_{\text{х1}}$,${\text{З}}_{\text{х2}}$ & задача & 1800 & 1800 & План пользователя \\

  \hline
  9 Объем выполняемых работ &
  ${\text{А}}_{\text{1}}$,${\text{А}}_{\text{2}}$ & задача & 1800 & 1800 & План пользователя \\

  \hline
  10 Средняя трудоемкость работ на задачу &
  ${\text{Т}}_{\text{С1}}$,${\text{Т}}_{\text{С2}}$ & Человеко-часов & 8 & 0.5 & Рас\-чи\-ты\-ва\-ет\-ся по данным пользователя \\

  \hline
  11 Количество часов работы в день &
  ${\text{Т}}_{\text{Ч}}$ & ч & 8 & 8 & Принято для рассчета \\

  \hline
  12 Ставка налога на прибыль &
  ${\text{Н}}_{\text{П}}$ & \% & & 18 & \\

  \hline
\end{longtable}


Экономия затрат на заработную плату в расчете на 1 задачу (${\text{C}}_{\text{зe}}$):
\begin{equation}
\label{sec:economics:effect:form_jew}
{\text{C}}_{\text{ЗЕ}} = \frac{ {\text{З}}_{\text{СМ}} \cdot ( {\text{Т}}_{\text{с1}} - {\text{Т}}_{\text{с2}} ) }{ {\text{Д}}_{\text{Р}} \cdot {\text{}Т}_{\text{Ч}} } \text{ ,}
\end{equation}
где ${\text{З}}_{\text{cм}}$ -- среднемесячная заработная плата одного программиста (руб.); ${\text{T}}_{\text{c1}}$, ${\text{T}}_{\text{c2}}$ -- снижение трудоемкости работ в расчете на 1 задачу (человеко-часов); ${\text{T}}_{\text{ч}}$ -- количество часов работы в день (ч); ${\text{Д}}_{\text{p}}$ -- среднемесячное количество рабочих дней.
$${\text{C}}_{\text{ЗЕ}} = 39.36 (\text{руб.})$$

Экономия заработной платы при использовании нового ПО (тыс руб.):
\begin{equation}
\label{sec:economics:effect:form_zp}
{\text{C}}_{\text{З}} = {\text{С}}_{\text{ЗЕ}} \cdot {\text{А}}_{\text{2}} \text{ ,}
\end{equation}
где ${\text{С}}_{\text{з}}$ -- экономия заработной платы; ${\text{А}}_{\text{2}}$ -- количество типовых задач, решаемых за год (задач).
$${\text{C}}_{\text{З}} = 70840 (\text{руб.})$$

Экономия с учетом начисления на зарплату (${\text{С}}_{\text{н}}$):
\begin{equation}
\label{sec:economics:effect:form_nach}
{\text{C}}_{\text{Н}} = {\text{С}}_{\text{З}} \cdot {\text{К}}_{\text{НЗ}}
\end{equation}
$${\text{C}}_{\text{Н}} = 85008 (\text{руб.})$$

Экономия за счет сокращения простоев сервиса (${\text{С}}_{\text{с}}$) рассчитывается по  формуле:
\begin{equation}
\label{sec:economics:effect:form_prost}
{\text{C}}_{\text{C}} = \frac{ ( {\text{П}}_{\text{1}} - {\text{П}}_{\text{2}} ) \cdot {\text{Д}}_{\text{РГ}} \cdot {\text{С}}_{\text{П}} }{ 60 } \text{ ,}
\end{equation}
где ${\text{Д}}_{\text{РГ}}$ -- плановый фонд работы сервиса (дней).
$${\text{C}}_{\text{C}} = 1158 (\text{руб.})$$

Общая готовая экономия текущих затрат, связанных с использованием нового ПО (${\text{С}}_{\text{O}}$), рассчитывается по формуле:
\begin{equation}
\label{sec:economics:effect:form_jew_big}
{\text{C}}_{\text{О}} = {\text{С}}_{\text{Н}} + {\text{С}}_{\text{С}}
\end{equation}
$${\text{C}}_{\text{О}} = 86167 (\text{руб.})$$

Внедрение нового ПО позволит пользователю сэкономить на текущих затратах, т.е. практически получить на эту сумму дополнительную прибыль. Для пользователя в качестве экономического эффекта выступает лишь чистая прибыль -- дополнительная прибыль, остающаяся в его распоряжении (${\text{ΔП}}_{\text{Ч}}$), которая определяется по формуле:
\begin{equation}
\label{sec:economics:effect:form_dpch}
{\text{ΔП}}_{\text{Ч}} = {\text{С}}_{\text{O}} - \frac{ {\text{С}}_{\text{О}} \cdot {\text{Н}}_{\text{П}} }{ 100 }
\end{equation}
$${\text{ΔП}}_{\text{Ч}} = 70657 (\text{руб.})$$

В процессе использования нового ПО чистая прибыль в конечном итоге
возмещает капитальные затраты. Однако полученные при этом суммы результатов (прибыли) и затрат (капитальных вложений) по годам приводят к единому времени -- расчетному году (за расчетный год принят 2017-й год) путем умножения результатов и затрат за каждый год на коэффициент дисконтирования \text{α}. В данном примере используются коэффициенты: 2017 г. -- 1, 2018-й -- 0.8696, 2019-й -- 0.7561, 2020 г. -- 0.6575. Все рассчитанные данные экономического эффекта сводятся в таблицу \ref{table:economics:cost:developers}.


\begin{longtable}{{
  |>{\raggedright}m{0.145\textwidth} |
  >{\centering}m{0.1\textwidth} |
  >{\centering}m{0.16\textwidth} |
  >{\centering}m{0.13\textwidth} |
  >{\centering}m{0.15\textwidth} |
  >{\centering\arraybackslash}m{0.13\textwidth}|
}}
  \caption{Расчет экономического эффекта от использования нового программного средства}
  \label{sec:economics:effect:tab_effect}

  \hline
  {\begin{center} По\-ка\-за\-те\-ли \end{center}} & Еди\-ни\-цы измерения & 2017 г. & 2018 г. & 2019 г. & 2020 г. \\
  \endfirsthead

  \caption*{Продолжение таблицы \ref{sec:economics:effect:tab_effect}}\\
  \hline
	\centering 1 & \centering 2 & \centering 3 & \centering 4 & \centering 5 & \centering\arraybackslash 6 \\
  \hline
  \endhead

  \hline
	\centering 1 & \centering 2 & \centering 3 & \centering 4 & \centering 5 & \centering\arraybackslash 6 \\

  \hline
  Результаты & & & & & \\

  \hline
  Прирост прибыли за счет экономии затрат (${\text{П}}_{\text{ч}}$) &
  руб. & & \num{70657} & \num{70657} & \num{70657} \\

  %\hline
  То же с учетом фактора времени &
  руб. & & \num{61143.32} & \num{53423.75} & \num{46456.97} \\

  \hline
  Затраты & & & & & \\

  \hline
  При\-об\-ре\-те\-ние ПО (${\text{К}}_{\text{пр}}$) &
  руб. & \num{126352.38} & & & \\

  \hline
  Со\-про\-вож\-де\-ние (${\text{К}}_{\text{c}}$) &
  руб. & \num{17132.53} & & & \\

  \hline
  Всего затрат &
  руб. & \num{143484.91} & & & \\

  \hline
  Эко\-но\-ми\-че\-ский эффект & & & & & \\

  \hline
  Пре\-вы\-ше\-ние результата над затратами &
  руб. & \num{-143484.91} & \num{61443.32} & \num{53423.75} & \num{46456.97} \\

  \hline
  То же с нарастающим итогом &
  руб. & \num{-143484.91} & \num{-82041.6} & \num{-28617.84} & \num{17839.14} \\

  \hline
  Коэф\-фи\-ци\-ент приведения &
  еди\-ни\-цы & \num{1} & \num{0.8696} & \num{0.7561} & \num{0.6575} \\

  \hline
\end{longtable}
