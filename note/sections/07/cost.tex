\subsection{Расчет сметы затрат и цены программного продукта}
\label{sec:economics:cost}

Исходные данные, которые будут использоваться при расчете сметы затрат, представлены в таблице~\ref{table:economics:cost:initial_data}.

\begin{table}[!ht]
\caption{Исходные данные}
\label{table:economics:cost:initial_data}
\centering
	\begin{tabular}{{
      |>{\raggedright}m{0.4\textwidth} |
	    >{\centering}m{0.15\textwidth} |
      >{\centering}m{0.15\textwidth} |
	    >{\centering\arraybackslash}m{0.152\textwidth} |
  }}

  \hline
	{\begin{center} Наименование показателя \end{center}} & Буквенные обозначения &	Единицы измерения & Количество \\

	\hline
	Коэффициент новизны & $К_{\textit{Н}}$ & единиц & 0.9 \\

	\hline
	Группа сложности & & единиц & 1 \\

	\hline
	Дополнительный коэффициент сложности & $К_{\textit{СЛ}}$ & единиц & 0.18 \\

	\hline
	Поправочный коэффициент, учитывающий использование типовых программ & $К_{\textit{Т}}$ & единиц & 0.8 \\

	\hline
	Установленная плановая продолжительность разработки & $Т_{\textit{Р}}$ & лет & 1 \\

	\hline
	Продолжительность рабочего дня & $Т_{\textit{Ч}}$ & Ч & 8 \\

	\hline
	Тарифная ставка первого разряда & $Т_{\textit{М1}}$ & руб & 265 \\

	\hline
	Коэффициент премирования & $К_{\textit{П}}$ & единиц & 1.2 \\

	\hline
	Норматив дополнительной заработной платы & $Н_{\textit{Д}}$ & \% & 20 \\

	\hline
	Отчисления в фонд социальной защиты населения & $З_{\textit{СЗ}}$ & \% & 34 \\

	\hline
	Отчисления в Белгосстрах & $Н_{\textit{НС}}$ & \% & 0.6 \\

	\hline
	Расходы на научных командировки & $Р_{\textit{НКИ}}$ & \% & 30 \\

	\hline
	Прочие прямые расходы & $П_{\textit{ЗИ}}$ & \% & 20 \\

	\hline
	\end{tabular}
\end{table}

Объем программного средства определяется путем подбора аналогов на основании классификации типов программного средства, каталога функций, которые постоянно обновляются и утверждаются в установленном порядке. На основании информации и функциях разрабатываемого программного средства по каталогу функций определяется объем функций. Объем программного средства определяется на основе нормативных данных, приведенных в таблице \ref{table:economics:cost:size}.

\begin{table}[!ht]
\caption{Характеристика функций и их объем}
\label{table:economics:cost:size}
\centering
	\begin{tabular}{{
      |>{\centering}m{0.12\textwidth} |
	    >{\raggedright}m{0.6\textwidth} |
	    >{\centering\arraybackslash}m{0.2\textwidth}|
  }}

  \hline
	\No{} функции & {\begin{center} Наименование (содержание) функции \end{center}} & Объём функции \\

	\hline
	101 & Организация ввода информации & \num{150} \\

	\hline
	102 & Контроль, предварительная обработка и ввод информации & \num{450} \\

	\hline
	111 & Управление вводом/выводом & \num{2400} \\

	\hline
	201 & Генерация структуры базы данных & \num{4300} \\

	\hline
	203 & Формирование баз данных & \num{2180} \\

	\hline
	204 & Обработка наборов и записей базы данных & \num{2670} \\

	\hline
	207 & Манипулирование данными & \num{9550} \\

	\hline
	305 & Обработка файлов & \num{720} \\

	\hline
	507 & Обеспечение интерфейса между компонентами & \num{970} \\

	\hline
	701 & Математическая статистика и прогнозирование & \num{9320} \\

	\hline
	 Итого: & & \num{32710} \\

	\hline
	\end{tabular}
\end{table}

Объем программного средства вычисляется по формуле:
\begin{equation}
\label{formula:economics:cost:f_vo}
V_{O} = \sum_{i=1}^n V_{i}
\end{equation}
где $V_{i}$ -- объем отдельной функции ПО; $n$ -- общее число функций; $V_{\text{О}}$ -- общий объем ПС.

$V_{\text{О}} = 32710$ (строк исходного кода).

Исходя из режима работы в реальном времени, а также обеспечения существенного распараллеливания вычислений и реализации особо сложных инженерных и научных расчетов, применяется коэффициент $K_{\text{С}}$ к объему ПО, который определяется по формуле:
\begin{equation}
\label{formula:economics:cost:f_kc}
K_{\text{С}} = 1 + \sum_{i=1}^n K_{i}
\end{equation}
где $K_{i}$ -- коэффициент, соответствующий степени повышения сложности ПО за счет конкретной характеристики;
$$K_{\text{C}} = 1 + 0.06 + 0.06 + 0.06 = 1.18$$.

С учетом дополнительного коэффициента сложности ${\text{К}}_{\text{СЛ}}$ рассчитывается общая трудоемкость ПС по формуле:
\begin{equation}
\label{formula:economics:cost:f_kcl}
{\text{T}}_{\text{O}} = {\text{T}}_{\text{Н}} + {\text{К}}_{\text{С}}
\end{equation}
где ${\text{Т}}_{\text{O}}$ -- общая трудоемкость; ${\text{T}}_{\text{H}}$ -- нормативная трудоемкость ПС; ${\text{K}}_{\text{C}}$ -- дополнительный коэффициент сложности ПС. Нормативная трудоемкость ПО (${\text{T}}_{\text{H}}$) -- 1087 чел./дн.; коэффициент сложности (${\text{K}}_{\text{C}}$) -- 1.12; коэффициент, учитывающий степень использования при разработке ПО стандартных модулей (${\text{K}}_{\text{T}}$) -- 0.8; коэффициент новизны разрабатываемого ПО (${\text{K}}_{\text{H}}$) -- 0.9.

Общая трудоемкость:
$${\text{T}}_{\text{O}} = {\text{T}}_{\text{H}} \cdot {\text{K}}_{\text{C}} \cdot {\text{K}}_{\text{T}} \cdot {\text{K}}_{\text{H}} = 830 \cdot 1.18 \cdot 0.8 \cdot 0.9 = 705 (\text{чел./дн.})$$.

На основе уточненной трудоемкости разработки ПС и установленного периода разработки, общая плановая численность разработчиком равна:
\begin{equation}
\label{formula:economics:cost:f_chr}
{\text{Ч}}_{\text{Р}} = \frac{ {\text{Т}}_{\text{О}} }{ {\text{Т}}_{\text{Р}} \cdot {\text{Ф}}_{\text{ЭФ}} }
\end{equation}
где ${\text{Ф}}_{\text{ЭФ}}$ -- эффективный фонд времени работы одного работника в течение года (дн.); ${\text{Т}}_{\text{О}}$ -- общая трудоемкость разработки проекта (чел./дн.); ${\text{Т}}_{\text{Р}}$ -- срок разработки проекта (лет).

Эффективный фонд времени работы одного работника (${\text{Ф}}_{\text{ЭФ}}$) рассчитывается по формуле:
\begin{equation}
\label{formula:economics:cost:f_fef}
{\text{Ф}}_{\text{ЭФ}} = {\text{Д}}_{\text{Г}} - {\text{Д}}_{\text{П}} - {\text{Д}}_{\text{В}} - {\text{Д}}_{\text{О}}
\end{equation}
где ${\text{Д}}_{\text{Г}}$ -- количество дней в году; ${\text{Д}}_{\text{П}}$ -- количество праздничных дней в году; ${\text{Д}}_{\text{В}}$ -- количество выходных дней в году; ${\text{Д}}_{\text{О}}$ -- количество дней отпуска.

${\text{Ф}}_{\text{ЭФ}} = 238$ дней в году.

Срок разработки установлен 12 месяцев (${\text{Т}}_{\text{Р}} = 1$ г
$${\text{Ч}}_{\text{Р}} = \frac{ 705 }{ 1.0 \cdot 238 } \approx 2.96 \approx 3$$.

\begin{table}[!ht]
  \caption{Расчет утонченной трудоемкости ПС и численности исполнителей по стадиям}
  \label{table:economics:cost:work}
  \centering
  \begin{tabular}{{
    |>{\raggedright}m{0.3\textwidth} |
    >{\raggedright}m{0.08\textwidth} |
    >{\raggedright}m{0.08\textwidth} |
    >{\raggedright}m{0.08\textwidth} |
    >{\raggedright}m{0.08\textwidth} |
    >{\raggedright}m{0.08\textwidth} |
    >{\raggedright\arraybackslash}m{0.08\textwidth}|
  }}

  \hline
  {\begin{centering} Показатели \end{centering}} & \multicolumn{5}{c|}{Стадии} &
  {\begin{centering} Итого \end{centering}} \\

  \hline
  & ТЗ & ЭП & ТП & РП & ВН & \\

  \hline
  1.Коэффициенты удельныхвесов трудоемкости стадии разработки ПО (d) &
  \num{0.10} & \num{0.08} & \num{0.09} & \num{0.58} & \num{0.15} & \num{0.10} \\

  \hline
  2.Коэффициент сложности ПО (${\text{K}}_{\text{С}}$) &
  \num{0.18} & \num{0.18} & \num{0.18} & \num{0.18} & \num{0.18} & \\

  \hline
  3.Коэффициент, учитывающий использование стандартных модулей & & & & \num{0.8} & & \\

  \hline
  4.Коэффициент, учитывающий новизну ПО (${\text{К}}_{\text{Н}}$) &
  \num{0.9} & \num{0.9} & \num{0.9} & \num{0.9} & \num{0.9} & \\

  \hline
  5.Численность исполнителей, чел. (${\text{Ч}}_{\text{И}}$) &
  \num{3} & \num{3} & \num{3} & \num{3} & \num{3} & \num{3} \\

  \hline
  6.Сроки разработки, лет & & & & & & 1 \\

  \hline
  \end{tabular}
\end{table}

Реализацией проекта занимались 3 человека. В соответствии с численностью и выполняемым функциями устанавливается штатное расписание \\группы специалистов-разработчиков.

Расчет основной заработной платы осуществляется в следующей последовательности. Определим месячные (${\text{Т}}_{\text{М}}$) и часовые (${\text{Т}}_{\text{Ч}}$) тарифные ставки начальника отдела (тарифный разряд -- 15; тарифный коэффициент – 3.48), инженера-программиста 1-й категории (тарифный разряд -- 14; тарифный коэффициент -- 3.25). Месячный тарифный оклад (${\text{Т}}_{\text{МО}}$) определяется путем умножения действующей месячной тарифной ставки 1-го разряда (${\text{Т}}_{\text{М1}}$) на тарифный коэффициент (${\text{Т}}_{\text{Ki}}$), соответствующий установленному тарифному разряду специалиста:
\begin{equation}
\label{formula:economics:cost:f_tmoi}
{\text{Т}}_{\text{МОi}} = {\text{T}}_{\text{М1}} \cdot {\text{Т}}_{\text{Кi}}
\end{equation}

Часовая тарифная ставка рассчитывается путем деления месячной тарифной ставки на установленный при сорокачасовой рабочей неделе в восьмичасовом рабочем дне фонд рабочего времени – 168 часов:
\begin{equation}
\label{formula:economics:cost:f_chts}
{\text{Т}}_{\text{Ч}} = {\text{Т}}_{\text{М}} / 168
\end{equation}
где ${\text{Т}}_{\text{Ч}}$ -- часовая тарифная ставка (руб); ${\text{Т}}_{\text{М}}$ -- месячная тарифная ставка(руб).

Месячная и часовая тарифные ставки начальника отдела:
$${\text{Т}}_{\text{м}} = 265 \cdot 3.48 = 768.5 (\text{руб})$$
$${\text{Т}}_{\text{ч}} = 768.5 / 168 = 4.58 (\text{руб})$$

Месячная и часовая тарифные ставки инженера-программиста 1-й категории равны соответственно:
$${\text{Т}}_{\text{м}} = 265 \cdot 3.25 = 717.7 (\text{руб})$$
$${\text{Т}}_{\text{ч}} = 717.7 / 168 = 4.27 (\text{руб})$$

Расчет месячных и почасовых тарифных ставок сведен в таблицу \ref{table:economics:cost:developers}.

\begin{table}[!ht]
  \caption{Штатное расписание группы разработчиков}
  \label{table:economics:cost:developers}
  \centering
  \begin{tabular}{{
    |>{\raggedright}m{0.14\textwidth} |
    >{\raggedright}m{0.14\textwidth} |
    >{\raggedright}m{0.14\textwidth} |
    >{\raggedright}m{0.14\textwidth} |
    >{\raggedright}m{0.14\textwidth} |
    >{\raggedright\arraybackslash}m{0.14\textwidth}|
  }}

  \hline
  {\begin{centering} Долж\-ность \end{centering}} &
  {\begin{centering} Ко\-ли\-чес\-тво ставок \end{centering}} &
  {\begin{centering} Тарифный разряд \end{centering}} &
  {\begin{centering} Тарифный коэффициент \end{centering}} &
  {\begin{centering} Месячная тарифная ставка (руб) \end{centering}} &
  {\begin{centering} Часовая тарифная ставка(руб) \end{centering}} \\

  \hline
  Начальник отдела (ведущий инженер программист) &
  \num{1.0} & \num{15} & \num{3.48} & \num{768.5} & \num{4.57} \\

  \hline
  Инженер-про\-грам\-мист 1-й категории &
  \num{1.0} & \num{14} & \num{3.25} & \num{717.7} & \num{4.27} \\

  \hline
  Инженер-про\-грам\-мист 1-й категории &
  \num{1.0} & \num{14} & \num{3.25} & \num{717.7} & \num{4.27} \\

  \hline
  \end{tabular}
\end{table}

Основная заработная плата исполнителей на конкретное ПО рассчитывается по формуле:
\begin{equation}
\label{formula:economics:cost:f_zp}
{\text{З}}_{\text{О}} = \summ_{i=1}^n {\text{З}}_{\text{Сi}} \cdot {\text{Ф}}_{\text{Рi}} \cdot {\text{К}}_{\text{i}}
\end{equation}
где $n$ -- количество исполнителей, занятых разработкой конкретного ПО; ${\text{З}}_{\text{Сi}}$ -- среднедневная заработная плата i-го исполнителя (д.е.); ${\text{Ф}}_{\text{Рi}}$ -- плановый фонд рабочего времени i-го исполнителя (дн.); ${\text{К}}$ -- коэффициент премирования ($1.2$).
$$4.57 \cdot 8 \cdot 238 \cdot 1.2 + 4.27 \cdot 8 \cdot 238 \cdot 1.2 + 4.27 \cdot 8 \cdot 238 \cdot 1.2 = 31736.4 (\text{руб})$$

Дополнительная заработная плата исполнителей проекта определяется по формуле:
\begin{equation}
\label{formula:economics:cost:f_add_zp}
{\text{З}}_{\text{Д}} = \frac{ {\text{З}}_{\text{О}} \cdot {\text{Н}}_{\text{Д}} }{ 100 }
\end{equation}

$$\frac{ 31736.4 \cdot 20 }{ 100 } = 6347.28 (\text{руб})$$

Отчисления в фонд социальной защиты населения и на обязательное страхование (${\text{З}}_{\text{С}}$) определяются в соответствии с действующими законодательными актами по формуле:
\begin{equation}
\label{formula:economics:cost:f_fszn}
{\text{З}}_{\text{СЗ}} = \frac{ ({\text{З}}_{\text{О}} + {\text{З}}_{\text{Д}}) \cdot {\text{Н}}_{\text{СЗ}} }{ 100 }
\end{equation}
где ${\text{Н}}_{\text{СЗ}}$ -- норматив отчислений в фонд социальной защиты населения и на обязательное страхование (34\% + 0.6\%).
$$\frac{ (31736.4 + 6347.28) \cdot 34.6 }{ 100 } = 13176.95 (\text{руб})$$

Расходы по статье «Машинное время» (${\text{Р}}_{\text{М}}$) включают оплату машинного времени, необходимого для разработки и отладки ПС, и определяются по формуле:
\begin{equation}
\label{formula:economics:cost:f_mt}
{\text{Р}}_{\text{М}} = {\text{Ц}}_{\text{М}} \cdot {\text{Т}}_{\text{Ч}} \cdot {\text{Т}}_{\text{Р}}
\end{equation}
$${\text{Р}}_{\text{М}} = 2665.6 (\text{руб})$$

Затраты по статье «Накладные расходы» (${\text{Р}}_{\text{Н}}$), связанные с необходимостью содержания аппарата управления, вспомогательных хозяйств и опытных (экспериментальных) производств, а также с расходами на общехозяйственные нужды (${\text{Р}}_{\text{Н}}$), определяются по формуле:
\begin{equation}
\label{formula:economics:cost:f_nr}
{\text{Р}}_{\text{Н}} = \frac{ {\text{З}}_{\text{СЗ}} \cdot {\text{Н}}_{\text{РН}} }{ 100 }
\end{equation}
$$\frac{ 40578.05 \cdot 100  }{ 100 } = 40578.05 (\text{руб})$$

Полная себестоимость:
\begin{equation}
\label{formula:economics:cost:f_psb}
{\text{С}}_{\text{П}} = {\text{З}}_{\text{О}} + {\text{З}}_{\text{Д}} + {\text{Р}}_{\text{М}} + {\text{З}}_{\text{СЗ}} + {\text{Р}}_{\text{Н}}
\end{equation}
$${\text{С}}_{\text{П}} = 85662.63 (\text{руб})$$

Прогнозируемая прибыль ПС рассчитывается по формуле:
\begin{equation}
\label{formula:economics:cost:f_pp}
{\text{П}}_{\text{ПС}} = \frac{ {\text{С}}_{\text{П}} \cdot {\text{У}}_{\text{Р}} }{ 100 }
\end{equation}
$$\frac{ 85662 \cdot 25  }{ 100 } = 21415.66 (\text{руб})$$

Прогнозируемая отпускная цена ПС вычисляется по формуле:
\begin{equation}
\label{formula:economics:cost:f_oprice}
{\text{Ц}}_{\text{П}} = {\text{С}}_{\text{П}} + {\text{П}}_{\text{ПС}}
\end{equation}
$${\text{Ц}}_{\text{П}} = 107078 (\text{руб})$$

Налог на добавленную стоимость (НДС):
\begin{equation}
\label{formula:economics:cost:f_nds}
{\text{НДС}} = \frac{ {\text{Ц}}_{\text{П}} \cdot {\text{Н}}_{\text{ДС}} }{ 100 }
\end{equation}
где ${\text{Н}}_{\text{ДС}}$ -- норматив НДС (\%).
$${\text{НДС}} = 19274.09 (\text{руб})$$

Отпускная цена:
\begin{equation}
\label{formula:economics:cost:f_price}
{\text{Ц}}_{\text{О}} = {\text{Ц}}_{\text{П}} + {\text{НДС}}
\end{equation}
$${\text{Ц}}_{\text{О}} = 126352 (\text{руб})$$

Кроме того, организация-разработчик осуществляет затраты на сопровождение ПС (${\text{Р}}_{\text{С}}$):
\begin{equation}
\label{formula:economics:cost:f_sup}
{\text{Р}}_{\text{С}} = \frac{ {\text{С}}_{\text{П}} \cdot {\text{Н}}_{\text{С}} }{ 100 }
\end{equation}
где ${\text{Н}}_{\text{С}}$ -- норматив расходов на сопровождение и адаптацию (20\%).
$${\text{Р}}_{\text{С}} = 17132.53 (\text{руб})$$
