\subsection{Вывод по технико-экономическому обоснованию}
\label{sec:economics:conclusion}
Исходя из расчётов затрат на разработку, можно сделать вывод, что для разработки сервиса для извлечения характеристик музыкальной композиции понадобится один год, в разработке участвуют три программиста. Полная себестоимость составляет 85662.63 руб., отпускная цена программного продукта с уровнем рентабельности 25\% - 126352 руб. Также организация-разработчик должна осуществить затраты на сопровождение ПС в размере 17132.53 руб.

Исходя из таблицы расчёта экономического эффекта использования сервиса для извлечения характеристик музыкальной композиции, можно прийти к выводу, что данный продукт позволяет существенно сократить затраты на анализ музыкальных композиций. Положительный экономический эффект заключается в значительном уменьшении трудоёмкости извлечения характеристик музыкальных композиций. Продукт является экономически выгодным благодаря тому, что его окупаемость достигается по истечении трех лет. В конце этого срока чистая прибыль составит 17839.14 рублей.
