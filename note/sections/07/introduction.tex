\subsection{Введение и исходные данные}
\label{sec:economics:introduction}

Целью дипломной работы является создание программного средства, которое позволит автоматизировать сбор характеристик музыкальных композиций. С помощью него значительно сокращается время сбора характеристик для дальнейшего анализа аудиокомпозиций. Автоматизация процессов сокращает риск возникновения человеческой ошибки, что повышает точность классификации и приводит к генерации более точных рекомендаций и увеличению аудитории медиасервисов.

Целью данного технико-экономического обоснования является определение экономической эффективности создания данного программного продукта и его дальнейшего применения. Экономическая эффективность рассчитывается у разработчика и пользователя.

Программный комплекс относится к 1-й группе сложности. Для оценки экономической эффективности разработанного программного средства проводится расчет сметы затрат и цены программного продукта, а также прибыли от продажи одной системы (программы). Расчеты выполнены на основе методического пособия \cite{palitsyn}.
