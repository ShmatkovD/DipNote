\sectioncentered*{Определения и сокращения}
\label{sec:definitions}

В настоящей пояснительной записке применяются следующие определения и сокращения.
\\

\emph{API} -- Application Programming Interface (программный интерфейс приложения) \cite{istqb_specification}.


\sectioncentered*{Введение}
\addcontentsline{toc}{section}{Введение}
\label{sec:introduction}


Подъем сервисов электронного распространения медиаинформации дал беспрецендентный доступ пользователям к записям музыкальных композиций. Сейчас такие сервисы, как iTunes, Google Music, Яндекс Музыка, Spotify и другие, предоставляют мгновенный доступ к миллионам записей. Перед пользователями возникает проблема выбора следующей композиции для прослушивания. Для решения этой проблемы, а так же для того, чтобы облегчить ориентацию в этом большом количестве информации, сервисы электронного распространения записей предоставляют пользователям системы рекомендации контента.

Современные системы рекомендации контента обычно используют статистику прослушивания композиций пользователями, чтобы сделать рекомендации более точными. Однако на такие системы часто оказывают влияние смещения популярности в больших масштабах, из-за чего системы не могут рекомендовать композиции, которые менее популярны. Так же это затрудняет создание индивидуальных рекомендаций для пользователей, исходя из их предпочтений. Поэтому возникает необходимость извлекать информацию из композиций, а так же классифицировать сами копозиции. На сегодняшний день извлечение характеристик и классификация музыкальных композиций является интересной и соревновательной задачей, которой занимаются различные компании и исследователи. На данный момент сильное развитие получают способы, основанные на машинном обучении.


Целью данного дипломного проекта является создание сервис для извлечения музыкальных характеристик, который будет пригоден для дальнейшей интеграцией с другими системами. В ходе работы предстоит выполнить следующие задачи:
\begin{enumerate}
  \item выбрать способ извлечения базовых характеристик музыкальных композиций (таких как спектр, ритм, продолжительность);
  \item построить модель машинного обучения для излечения высокоуровневых характеристик;
  \item собрать и упорядочить данные для обучения модели;
  \item построить эффективную архитектуру сервиса;
  \item построить API для интеграции с другими системами.
\end{enumerate}

Реализация сервиса для извлечения характеристик музыкальных композиций позволит упростить задачу классификации контента для сервисов электронного распространения записей, а так же создать систему для создания более точных и индивидуальных рекомендаций для пользователей.
