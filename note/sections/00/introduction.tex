\sectioncentered*{Определения и сокращения}
\label{sec:definitions}

В настоящей пояснительной записке применяются следующие определения и сокращения.
\\

\emph{API} -- Application Programming Interface (программный интерфейс приложения).

\emph{t-SNE} -- алгоритм машинного обучения, который относят к методам множественного обучения признаков \cite{tsne}.

\emph{music2vec} -- английское сокращение от music to vector (композиция в вектор).

\emph{MP3} -- кодек третьего уровня, разработанный командой MPEG, формат файла для хранения аудиоинформации.

\emph{WAV} -- формат файла-контейнера для хранения записи оцифрованного аудиопотока.

\emph{FLAC} -- популярный свободный кодек, предназначенный для сжатия аудиоданных без потерь.

\emph{M4A} -- проприетарный (патентованный) формат аудиофайла с потерями.

\emph{JSON} -- JavaScript Object Notation (текстовый формат обмена данными, основанный на JavaScript).

\emph{XML} -- eXtensible Markup Language (расширяемый язык разметки).

\emph{JVM} -- Java Virtual Machine (основная часть исполняющей системы \\Java).

\emph{MSIL} -- Microsoft Intermediate Language (внутренний системный язык программирования Microsoft).

\emph{LLVM} -- Low Level Virtual Machine (универсальная система анализа, трансформации и оптимизации программ).

\emph{JIT} -- динамическая компиляция, технология увеличения производительности программных систем, использующих байт-код, путём компиляции байт-кода в машинный код или в другой формат непосредственно во время работы программы.

\emph{PyPy} -- интерпретатор языка программирования Python.

\emph{ANSI} -- американский национальный институт стандартов.

\emph{ISO} -- международная организация по стандартизации.

\emph{TPU} -- специализированные интегральные схемы, разработанные специально для машинного обучения.

\emph{ASIC} -- интегральная схема, специализированная для решения конкретной задачи.

\emph{ИИ} -- искусственный интеллект.

\emph{cgroups} -- механизм ядра Linux, который ограничивает и изолирует вычислительные ресурсы (процессорные, сетевые, ресурсы памяти, ресурсы \\ввода-вывода) для групп процессов.

\emph{LXC} -- система виртуализации на уровне операционной системы для запуска нескольких изолированных экземпляров операционной системы Linux на одном узле.

\emph{HTTP} -- HyperText Transfer Protocol (протокол прикладного уровня передачи данных).

\emph{WSGI} -- Web Server Gateway Interface (стандарт взаимодействия между Python-программой, выполняющейся на стороне сервера, и самим веб-сервером).

\emph{UNIX} -- семейство переносимых, многозадачных и многопользовательских операционных систем.

\emph{fork} -- системный вызов, создающий новый процесс (потомок), который является практически полной копией процесса-родителя, выполняющего этот вызов.

\emph{exec} -- системный вызов, загружающий в пространство процесса новую программу.

\emph{Affero GPLv3} -- свободная лицензия, созданная специально для таких программ, как веб‐приложения, так что пользователи, использующие изменённую программу через сеть, могут получить её исходный код.

\emph{Фреймворк} -- программная платформа, определяющая структуру программной системы; программное обеспечение, облегчающее разработку и \\объе\-ди\-не\-ние разных компонентов большого программного проекта.

\emph{RNN} -- рекуррентная нейронная сеть.

\emph{Linux} -- семейство Unix-подобных операционных систем на базе ядра Linux, включающих тот или иной набор утилит и программ проекта GNU, и, возможно, другие компоненты.

\emph{Debian} -- операционная система, основанная на Linux, состоящая из свободного ПО с открытым исходным кодом.


\sectioncentered*{Введение}
\addcontentsline{toc}{section}{Введение}
\label{sec:introduction}


Подъем попурярности сервисов электронного распространения медиаинформации дал беспрецендентный доступ пользователям к записям музыкальных композиций. Сейчас такие сервисы, как iTunes, Google Music, Яндекс Музыка, Spotify и другие, предоставляют мгновенный доступ к миллионам записей. Перед пользователями возникает проблема выбора следующей композиции для прослушивания. Для решения этой проблемы, а так же для того, чтобы облегчить ориентацию в этом большом количестве информации, сервисы электронного распространения записей предоставляют пользователям системы рекомендации контента.

Современные системы рекомендации контента обычно используют статистику прослушивания композиций пользователями, чтобы сделать рекомендации более точными. Однако на такие системы часто оказывают влияние смещения популярности в больших масштабах, из-за чего системы не могут рекомендовать композиции, которые менее популярны. Так же это затрудняет создание индивидуальных рекомендаций для пользователей, исходя из их предпочтений. Поэтому возникает необходимость извлекать информацию из композиций, а так же классифицировать сами копозиции. На сегодняшний день извлечение характеристик и классификация музыкальных композиций является интересной и соревновательной задачей, которой занимаются различные компании и исследователи. На данный момент сильное развитие получают способы, основанные на машинном обучении.


Целью данного дипломного проекта является создание сервис для извлечения музыкальных характеристик, который будет пригоден для дальнейшей интеграцией с другими системами. В ходе работы предстоит выполнить следующие задачи:
\begin{enumerate}
  \item выбрать способ извлечения базовых характеристик музыкальных \linebreak ком\-по\-зи\-ций (таких как спектр, ритм, продолжительность);
  \item построить модель машинного обучения для излечения высокоуровневых характеристик;
  \item собрать и упорядочить данные для обучения модели;
  \item построить эффективную архитектуру сервиса;
  \item построить API для интеграции с другими системами.
\end{enumerate}

Реализация сервиса для извлечения характеристик музыкальных композиций позволит упростить задачу классификации контента для сервисов электронного распространения записей, а так же создать систему для создания более точных и индивидуальных рекомендаций для пользователей.
