\sectioncentered*{Заключение}
\addcontentsline{toc}{section}{Заключение}

Данный дипломный проект относится к области извлечения музыкальной информации. Был проведен анализ продуктов, которые являются аналогами реализуемого проекта, а так же предметной области в целом. По результатам анализа был сделан вывод, что на данный момент в этой области ведуться активные исследования и разработки, так же были оценены аналоги, выявлены их достоинства и недостатки.

На основании проведенного анализа предметной области были выдвинуты требования к программному средству. В качестве технолгогий разработки были выбраны наиболее современные средства, применяемые для решения подобных задач. Спроектированное средство показало высокую точность классификации композиций по жанрам, что показывает эффективность выбранного подхода. Расширение распространения автоматизации на извлечение музыкальных характеристик высокого уровня позволяет сделать вывод о целесообразности разработки данного программного средства. Это было подтверждено так же в ходе выполнения экономического обоснования.

В итоге было разработано пограммное средство, позволяющее автоматизированно извлекать характеристики музыкальных композиций высокого уровня.

В дальнейшем планируется увеличивать точность извлечения характеристик. Так же планируется применить подходы, которые не требуют тренировки и способны самостоятельно обучаться извлекать характеристики в процессе работы. Кроме извлекаемых, существуют другие характеристики композиций, поддержку которых планируется внедрить в данное программное средство.
