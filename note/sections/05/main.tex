\section{Тренировка моделей машинного обучения}
\label{sec:training}

В нашем случае необходимо построить классификатор, который будет создавать соответствие между музыкальной композицией и определенным музыкальным жанром. Количество жанров будет ограничено, так как набор жанров известен заранее. Для такого типа классификаторов наилучшим образом подходит обучение с учителем.

Обучение с учителем — один из способов машинного обучения, в ходе которого испытуемая система принудительно обучается с помощью примеров «стимул-реакция». С точки зрения кибернетики, является одним из видов кибернетического эксперимента. Между входами и эталонными выходами (стимул-реакция) может существовать некоторая зависимость, но она неизвестна. Известна только конечная совокупность прецедентов — пар «стимул-реакция», называемая обучающей выборкой. На основе этих данных требуется восстановить зависимость (построить модель отношений стимул-реакция, пригодных для прогнозирования), то есть построить алгоритм, способный для любого объекта выдать достаточно точный ответ. Для измерения точности ответов, так же как и в обучении на примерах, может вводиться функционал качества.

Для того, чтобы обучить модель, необходимо обладать достаточным количеством данных для обучения. В качестве данных используются заранее размеченные композиции, т.е. композиции с известным жанром. Композиции, на которых будет обучаться модель, являются композициями типичных представителей или основателей музыкальных направлений. Всего для обучения используется около одинадцати тысяч записей.

Для того, чтобы оптимизировать обучение, композиции были подвергнуты первичной обработке. Первичная обработка состояла в том, чтобы преобразовать композиции в векторы, которые будут поступать на вход модели. Для обработки использовался метод, который преобразовывает композицию в набор мел-частотных кепстральных коэффициентов, описанный в части \ref{sec:design:dev}.

Для обучения выборка разбивается на несколько фрагментов. Выборка разделяется на непересекающиеся подмножества элементов, на которых будет происходить обучение. В нашем случае мы будем разбивать выборку на 3 подмножества:
\begin{enumerate}
  \item обучающее;
  \item валидирующее;
  \item тестирующее.
\end{enumerate}

Элементы обучающего подмножества будут использоваться для того, чтобы проверять реакцию модели и, соответственно, корректировать веса так, чтобы повышать точность классификации.

Валидирующее подмножество так же используется для обучения модели. Это случаи, на которых не проверялась реакция модели. Они необходимы для того, чтобы модель не переобучилась определять жанр только для элементов обучающего подмножества. Это внесет поправку в алгоритм обучения, которая зависит только от значений метрики точности, что позволит использовать эти данные для валидации в дальнейшем.

Элементы тестирующего множества необходимы только для того, чтобы проверить итоговую точность построенных моделью предсказаний.

Обучение разделено на эпохи. Каждая эпоха состоит из нескольких шагов:
\begin{enumerate}
  \item обучение на обучающей выборке в прямом порядке;
  \item обучение на обучающей выборке в обратном порядке;
  \item проверка на переобучение с помощью валидирующей выборки;
  \item перемешивание элементов в обучающей выборке;
  \item перемешивание элементов в валидирующей выборке.
\end{enumerate}

Перемешивание элементов необходимо для того, чтобы исключить \linebreak случайный эффект, когда модель начнет переобучаться из-за того, что в нее будут поступать циклические данные.

Количество эпох для тренировки модели вычисляется эмпирически. Модели построены так, чтобы иметь возможность их дообучить при необходимости.

Для обучения моделей необходимы ресурсы, которые значительно превышают ресурсы, которые необходимы для работы моделей в обычном режиме.
