\subsection{Требования к проектируемому программному средству}
\label{sec:design:specification}

По результатам изучения предметной области, анализа литературных источников и обзора существующих систем-аналогов сформулируем требования к проектируемому программному средству.

Сервис должен предоставлять простой API для извлечения характеристик из аудиозаписей. Для поддержки большего числа пользователей и эффективного процесса интеграции с системами пользователя сервису необходимо поддерживать основные форматы взаимодействия между сервисами:
\begin{enumerate}
  \item JSON;
  \item XML.
\end{enumerate}

Сервис будет являться модулем, который достаточно просто встраивать в существующие системы. Для повышения удобства использования развертывание модуля должно быть простым процессом, который не занимает много времени.

Для подготовки композиций к дальнейшему анализу, а так же для их анализа, в сервисе реализованы сложные математические вычисления. Для того, чтобы повысить комфортность использования, сервису следует анализировать композиции за приемлемое время. Так же необходимо снизить требовательность к характеристикам компонентов вычислительной системы, на которой будет запущен сервис. Поэтому необходимо оптимизировать вычисления, производимые для анализа композиций.

Музыкальные жанры подвержены изменению. Для того, чтобы поддерживать аналитическую составляющую в актуальном состоянии, необходимо иметь возможность обновлять части системы. Поэтому сервис должен быть построен из независимых компонентов, согласованных на уровне интерфейсов.

Аудиозаписи могут храниться в различных форматах, поэтому необходимо реализовать поддержку обработки основных форматов распростанения цифровых аудиозаписей, таких как mp3, wav, flac, m4a.

Для предоставления как можно более полной информации об аудиозаписи извлекаемые характеристики должны быть различных уровней, т.е. высокоуровневые и низкоуровневые. Определим основные характеристики композиции для извлечения:
\begin{enumerate}
  \item длина;
  \item громкость;
  \item количество ударов в минуту;
  \item жанр.
\end{enumerate}

Так же необходимо учитывать, что человеческий слух воспринимает информацию несколько иначе, чем компьютер. Поэтому нужно делать поправку на особенности человеческого слуха при анализе аудиозаписей.
