\section{Требования к проектируемому программному средству}
\label{sec:specification}

По результатам изучения предметной области, анализа литературных источников и обзора существующих систем-аналогов сформулируем требования к проектируемому программному средству.

Сервис должен уметь извлекать информацию из распространенных форматов аудиозаписей.

Сервис должен выдавать высокоуровневые характеристики (жанр) и низкоуровневые характеристики (длина, частота ударов в минуту и тп).

Давать доступ к функциональности выше по API.

Так как в сервисе реализованы сложные математические вычисления, то они должны иметь достаточный уровень оптимальности, чтобы обрабатывать данные за разумное время с использованием разумного количества ресурсов. Так же уровень опимизации влияет на системные требования данного продукта.

Иметь модульную архитектуру для удобного изменения алгоритмов обработки информации на более актуальные.

Являтся модулем, который просто встраивается в сторонние системы.

Иметь адаптивный формат данных для более удобной интеграции с системой пользователя. -- пока не точно!!!

Обрабатывать звуковую информацию на уровне человеческого слуха. (в плане характеристик звука)

УПОРЯДОЧИТЬ!!!!
