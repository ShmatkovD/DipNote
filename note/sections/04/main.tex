\section{Проектирование программного средства}
\label{sec:design}

Для модульности - упаковать в контейнер - Docker

Для того, чтобы писать модели - Keras

Для того, чтобы модели исполнялись оптимизированно - TensorFlow

В качестве веб-сервера, чтобы было ничего лишнего - Flask

Для того, чтобы обеспечить общение между моделью и сервером - RedisMQ

Модель упаковать, чтобы все время висела в памяти - Daemon

Чтобы все запускать единомоментно - Supervisor

Чтобы считать спектры оптимальным образом - Essentia

Чтобы обрабатывать большое число запросов в секунду - Gunicorn

Чтобы задавть количество workers, default формат для общения - Environment variables???

СДЕЛАТЬ КРАСИВО И БОЛЬШЕ ВОДЫ!!!!

ВОЗМОЖНО ПЕРЕНЕСТИ ТО, ЧТО НИЖЕ К ПРОЕКТИРОВАНИЮ!!!

Для того, чтобы уменьшить количество обрабатываемой информации было решено использовать мел-частотные кепстральные коэффициенты. Достоинствами этого метода являются:
\begin{enumerate}
  \item Используется спектр сигнала, что позволяет учесть волновую природу звука;
  \item Спектр проецируется на мел-шкалу, что позволяет учесть восприятие частот человеческим ухом;
  \item Позволяет сжать количество информации количеством вычисляемых коэффициентов.
\end{enumerate}

Ниже приведен алгоритм вычесления мел-частотных кепстральных коэффициентов:
\begin{enumerate}
  \item Применить к сигналу оконное преобразование Фурье;
  \item Для каждого полученного спектра:
    \begin{enumerate}
      \item Расположить спектр на мел-шкале (в результате получим мел-частотный спектр);
      \item Возвести полученные коэффициенты в квадрат и прологорифмировать их;
      \item Применить к результату преобразование Фурье.
    \end{enumerate}
\end{enumerate}
