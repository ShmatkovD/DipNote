\section{Используемые технологии}
\label{sec:development}

\subsection{Essentia}
\label{sec:development:essentia}

Essentia - это библиотека C ++ с открытым исходным кодом, обладающая привязками Python, предназначенная для аудиоанализа. Она выпускается под лицензией Affero GPLv3 и также доступна под собственной лицензией по запросу. Библиотека содержит обширную коллекцию многократно используемых алгоритмов, реализующих функциональность ввода и вывода звука, стандартные блоки цифровой обработки сигналов, статистическую характеристику данных и большой набор спектральных, временных, тональных музыкальных обработчиков.

Essentia - это не фреймворк, а скорее набор алгоритмов, завернутый в библиотеку. Он сконструирован с упором на надежность, производительность и оптимальность предоставляемых алгоритмов, включая скорость вычислений и использование памяти, а также простоту использования. Поток анализа определяется и реализуется пользователем, в то время как Essentia заботится о деталях реализации используемых алгоритмов. Существует специальный режим потоковой передачи, в котором возможно соединять алгоритмы и запускать их автоматически, вместо того, чтобы явно указывать порядок выполнения с преимуществом менее стандартного кода и меньшим потреблением памяти. Ряд примеров предоставляется библиотекой, однако их не следует рассматривать как единственный правильный способ делать вещи. Большая часть алгоритмов Essentia хорошо подходит для приложений реального времени.

Предоставляемые функциональные возможности легко расширяемы и позволяют проводить как исследовательские эксперименты, так и разработку крупномасштабных промышленных приложений.

\subsection{Python}
\label{sec:development:python}

Python — высокоуровневый язык программирования общего назначения, ориентированный на повышение производительности разработчика и читаемости кода. Синтаксис ядра Python минималистичен. В то же время стандартная библиотека включает большой объём полезных функций.

Python поддерживает несколько парадигм программирования, в том числе структурное, объектно-ориентированное, функциональное, императивное и аспектно-ориентированное. Основные архитектурные черты — динамическая типизация, автоматическое управление памятью, полная интроспекция, механизм обработки исключений, поддержка многопоточных вычислений и удобные высокоуровневые структуры данных. Код в Python организовывается в функции и классы, которые могут объединяться в модули (они в свою очередь могут быть объединены в пакеты).

Эталонной реализацией Python является интерпретатор CPython, поддерживающий большинство активно используемых платформ. Он распространяется под свободной лицензией Python Software Foundation License, позволяющей использовать его без ограничений в любых приложениях, включая проприетарные. Есть реализации интерпретаторов для JVM (с возможностью компиляции), MSIL (с возможностью компиляции), LLVM и других. Проект PyPy предлагает реализацию Python с использованием JIT-компиляции, которая значительно увеличивает скорость выполнения Python-программ.

Python — активно развивающийся язык программирования, новые версии (с добавлением/изменением языковых свойств) выходят примерно раз в два с половиной года. Вследствие этого и некоторых других причин на Python отсутствуют стандарт ANSI, ISO или другие официальные стандарты, их роль выполняет CPython

\subsection{TensorFlow}
\label{sec:development:tensorflow}

TensorFlow - это библиотека программного обеспечения с открытым исходным кодом для машинного обучения по целому ряду задач и разработанная Google для удовлетворения своих потребностей в системах, способных создавать и обучать нейронные сети для обнаружения и расшифровки паттернов и корреляций, аналогичных обучению и рассуждениях, которые используют люди. В настоящее время он используется для исследований и производства продуктов Google, часто заменяя роль предшественника с закрытым исходным кодом, DistBelief. TensorFlow была первоначально разработана командой Google Brain для внутреннего использования Google, прежде чем выпустить ее под лицензией с открытым исходным кодом Apache 2.0 9 ноября 2015 года.

Начиная с 2011 года, Google Brain построила DistBelief как проприетарную систему машинного обучения, основанную на глубоком обучении нейронных сетей. Его использование быстро развивалось в различных компаниях родительской компании Alphabet как в исследовательских, так и в коммерческих целях. Google назначил нескольких компьютерных ученых, включая Джеффа Дина, для упрощения и реорганизации базы кода DistBelief в более быструю и более надежную библиотеку прикладного уровня, которая стала TensorFlow. В 2009 году команда во главе с Джеффри Хинтоном реализовала обобщенное обратное распространение и другие улучшения, которые позволили создать нейронные сети с существенно большей точностью, например, на 25\% сократить ошибки распознавания речи.

TensorFlow - это система машинного обучения второго поколения Google Brain, выпущенная как программное обеспечение с открытым исходным кодом 9 ноября 2015 года. Хотя эталонная реализация выполняется на отдельных устройствах, TensorFlow может работать на нескольких процессорах и графических процессорах (с дополнительными расширениями CUDA для универсальных вычислений на графических процессорах). TensorFlow доступен на 64-битных Linux, macOS и мобильных вычислительных платформах, включая Android и iOS.

Вычисления TensorFlow выражаются в виде графов прохождения данных с состоянием. Название TensorFlow происходит от операций, которые такие нейронные сети выполняют с многомерными массивами данных. Эти многомерные массивы называются тензорами. В июне 2016 года Джефф Дин из Google заявил, что 1500 репозиториев на GitHub упомянули TensorFlow, из которых только 5 пренадлежали Google.

В мае 2016 года Google объявил о своем тензорном процессоре (TPU) - специализированной ASIC, разработанной специально для машинного обучения и предназначенной для TensorFlow. TPU представляет собой программируемый ускоритель ИИ, предназначенный для обеспечения высокой пропускной способности арифметики с низкой точностью (например, 8-разрядной) и ориентированной на использование или запуск моделей, а не на их обучение. Google объявила, что они уже более года работают с моделями машинного обучения на основе TPU в своих центрах обработки данных, и обнаружили, что TPU обеспечивает на порядок более оптимизированную производительность на ватт для машинного обучения.

\subsection{Keras}
\label{sec:development:keras}

Keras - это высокоуровневый API нейронных сетей, написанный на Python и способный работать поверх TensorFlow или Theano. Он был разработан с упором на быстрое проведение экспериментов. Способность идти от идеи к результату с наименьшей задержкой является ключевой особенностью данной библиотеки.

Особенности Keras:
\begin{enumerate}
  \item возможность легко и быстро создавать прототипы (благодаря удобству пользователя, модульности и расширяемости);
  \item поддержка как сверточных нейронных сетей, так и рекуррентных нейронных сетей, а также их комбинаций;
  \item работа без проблем на процессоре и графическом процессоре.
\end{enumerate}

Руководящие принципы:
\begin{enumerate}
  \item удобство для пользователя (Keras - это API, предназначенный для людей, а не для машин, это ставит удобство использования в центр внимания и обеспечивает ясную обратную связь из-за пользовательской ошибки);
  \item модульность (под моделью понимается последовательность или граф автономных, полностью конфигурируемых модулей, которые могут быть подключены вместе с минимальными ограничениями);
  \item легкая расширяемость (новые модули легко добавлять, как новые классы и функции, а существующие модули предоставляют достаточно примеров, чтобы иметь возможность легко создавать новые модули);
  \item работа с Python (нет отдельных конфигурационных файлов моделей в декларативном формате, модели описаны в коде Python, который является компактным, более легким для отладки и позволяет легко расширяться).
\end{enumerate}

\subsection{Flask}
\label{sec:development:flask}

Flask - это небольшой фреймворк, написанный на языке Python, c весьма большим сообществом и множеством модулей на все случаи жизни. В отличии от, скажем, Django, Flask не навязывает определенное решение той или иной задачи. Вместо этого, он предлагает использовать различные сторонние или собственные решения по вашему усмотрению.

Одним из проектных решений во Flask является то, что простые задачи должны быть простыми; они не должны занимать много кода, и это не должно ограничивать вас. Поэтому было сделано несколько вариантов дизайна, некоторые люди могут посчитать это удивительным и необщепринятым. Например, Flask использует локальные треды внутри объектов, так что вы не должны передавать объекты в пределах одного запроса от функции к функции, оставаясь в безопасном треде. Хотя этот очень простой подход и позволяет сэкономить много времени, это также может вызвать некоторые проблемы для очень больших приложений, поскольку изменения в этих локальных объектах-потоках может произойти где угодно в том же потоке. Для того, чтобы решить эти проблемы, разработчики не скрывают от вас локальные объекты-потоки, вместо этого охватывают их и предоставляем вам много инструментов, чтобы сделать работу с ними настолько приятным насколько это возможно.

\subsection{Gunicorn}
\label{sec:development:gunicorn}

Gunicorn - это HTTP-сервер Python WSGI для UNIX. Это модель рабочих процессов, перенесенная из проекта Ruby Unicorn. Сервер Gunicorn в целом совместим с различными веб-фреймворками, прост в использовании, не нагружает ресурсы сервера и является достаточно быстрым.

Особенности:
\begin{enumerate}
  \item поддерживает WSGI, web2py, Django и Paster;
  \item автоматическое управление рабочими процессами;
  \item простая конфигурация Python;
  \item несколько рабочих конфигураций;
  \item различные серверные триггеры для расширяемости;
  \item совместимость с Python версий 2.6 и выше, а также Python версий 3.2 и выше.
\end{enumerate}

\subsection{Supervisor}
\label{sec:development:supervisor}

Supervisor - это система клиент-сервер, которая позволяет своим пользователям управлять несколькими процессами в UNIX-подобных операционных системах.

Часто сложно получить точный статус (запущен или не запущен) в процессах в UNIX. Файлы с идентификаторами процессов часто не соответствуют действительности. Supervisord запускает процессы как подпроцессы, поэтому он всегда знает истинный статус своих подпроцессов, и предоставляет возможность узнать статусы для его дочерних процессов.

Supervisor предоставляет вам одно место для запуска, остановки и мониторинга процессов. Процессами можно управлять индивидуально или в группах. Вы можете настроить Supervisor для предоставления локальной или удаленной командной строки и веб-интерфейса.

Supervisor запускает свои подпроцессы через fork/exec, а подпроцессы не запускает как демоны. Операционная система сразу же сигнализирует Supervisor, когда процесс завершается, в отличие от некоторых решений, которые полагаются на неточные файлы идентификаторов процессов и периодический опрос процессов, чтобы перезапустить завершившиеся процессы.

\subsection{Docker}
\label{sec:development:docker}

Docker -- программное обеспечение для автоматизации развёртывания и управления приложениями в среде виртуализации на уровне операционной системы. Позволяет поместить приложение со всем его окружением и зависимостями в контейнер, который может быть перенесен на любую Linux-систему с поддержкой cgroups в ядре, а также предоставляет среду по управлению контейнерами. Изначально использовал возможности LXC, с 2015 года применял собственную библиотеку, абстрагирующую виртуализационные возможности ядра Linux — libcontainer. С появлением ​Open Container Initiative начался переход от монолитной к модульной архитектуре.

Цели использования Docker:
\begin{enumerate}
  \item абстрагирование хост-системы от контейнеризованных приложений;
  \item простота масштабирования;
  \item простота управления зависимостями и версиями приложения;
  \item чрезвычайно легкие, изолированные среды выполнения;
  \item совместно используемые слои;
  \item возможность компоновки и предсказуемость.
\end{enumerate}

Приложения, реализующие этот подход к проектированию, должны иметь следующие характеристики:
\begin{enumerate}
  \item они не должны полагаться на особенности хост-системы;
  \item каждый компонент должен предоставлять консистентный API, который пользователи могут использовать для доступа к сервису;
  \item каждый сервис должен принимать во внимание переменные окружения в процессе первоначальной настройки;
  \item данные приложения должны храниться вне контейнера на примонтированных томах или в отдельных контейнерах с данными.
\end{enumerate}
